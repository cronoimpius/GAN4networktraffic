\chapter{Input Data}
To make new data similar from some existing ones, we need a dataset of how those data needs to look like.
For this reason we describe di input data, from which we can create new examples.
The inital data needs to be transformed in some way, so it can be given as an input of our GAN.
\\\\
The composition of our dataset is a pcap file. This kind of files contain packet data of a network and are
ofen used to analyze the network characteristics and detect if there're anomalies and intrusions.
Since this kind of data is difficult to manage, even if it's well formatted, a first step is to convert the 
data as a csv dataset.
\\\\
In this way we can easily manage the data before giving them to the network and viewing the final results in a
simpler way.
The pcap file is something like the following:
%image of an extraction of the pcap file
while the output of the conversion of the pcap into the csv is a dataset like:
%image of extraction of the dataset
\section{Data Conversion}
To convert the pcap file into the csv, and viceversa we have created the following script that aims to 
do this kind of transofrmations.
%extract of the script
The foundamental informations are extracted using Scapy \cite{scapy}, a python library used to extract data from packets
saved into a pcap file. The main information that we need are:
\begin{itemize}
    \item the packet number
    \item the protocol used 
    \item the timestamp
    \item the source ip and port
    \item the destination ip and port
    \item the flag if tcp protocol is used
\end{itemize}
from the script we extract those infos from the pcap and store them into the final csv, for each packet into 
the file.
\\\\
Once adjusted the data source, we can continue developing the core of the project: the GAN that generates new
data.
