\chapter*{Introduction}
Packet caputere dataset can be used for evaluating network-based detection-systems, also known as NIDS.
In this reasearch we want to generate new packet capture dataset that are realistic. 
\\\\
The approach used is based on \textit{Generative Adversial Networks (GANs)} which achieve good results for 
image generation. The main problem here is that work very well with continuos attributes, and they can 
process only them. However, packet capture data contain categorical attributes such IP addresses or port numbers.
Exploiting this characteristic of the data that we use, we can implement an approach for generating packet-capture
data to transform them into continuos values.
\\\\
Once the data is generated, there is a testing phase that also include the usage of a real NIDS to 
verify if they are corretly generated and identified. In that phase we use \textbf{ZEEK} that is an open 
source network analysis framework, but in our scope we use it mainly as a Network Intrusion Detection System.
\\\\
In the following pages we will discuss all the detail and the motivation for all the deciosion that has been
made.
